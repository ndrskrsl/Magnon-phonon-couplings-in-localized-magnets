%\documentclass[checkin,prb,groupedaddress]{revtex4}

%\documentclass[aps,prb,preprint,groupedaddress]{revtex4}
%\documentclass[aps,prb,preprint,superscriptaddress]{revtex4}
\documentclass[aps,prl,twocolumn,superscriptaddress,amsmath,amssymb,floats,aps,10pt]{revtex4-1}
%\bibliographystyle{apsrev}
\pdfoutput=1
\usepackage{graphicx}
\newcommand{\vL}{\mbox{\bf L}}
\newcommand{\vB}{\mbox{\bf B}}
\usepackage{bm}% bold math


\begin{document}


\title{Magnon-phonon couplings in localized magnets}% Force line breaks with \\

\author{S.L. Holm}
\email[]{sonja@fys.ku.dk}
\author{A. Kreisel}
\author{T.K. Sch\"affer}
\author{N. Momsen}
\author{A. Bakke}
\author{M. Bertelsen}
\author{P.J. Ray}
\affiliation{Nanoscience Center, Niels Bohr Institute, University of Copenhagen, 2100 Copenhagen {\O}, Denmark}

\author{J. Larsen}
\affiliation{Institute of Physics, Technical University of Denmark, 2800 Lyngby, Denmark}

\author{Z. Yamani}
\affiliation{Chalk River National Laboratories, Canada}

\author{J.O. Birk}
\author{U. Stuhr}
\author{Ch. Niedermayer}
\affiliation{Laboratory of Neutron Scattering, Paul Scherrer Institute, Villigen, Switzerland}

\author{P.P. Deen}
\affiliation{European Spallation Source, Stora Algatan 4, 22350 Lund, Sweden}

\author{A. Fennell}
\affiliation{Laboratory of Neutron Scattering, Paul Scherrer Institute, Villigen, Switzerland}

\author{B.M. Andersen}
\author{K. Lefmann}
\affiliation{Nanoscience Center, Niels Bohr Institute, University of Copenhagen, 2100 Copenhagen {\O}, Denmark}



\date{\today}% It is always \today, today,
%          \setcounter{equation}{0}
%  \renewcommand{\thefigure}{\Roman{figure}}
%   \renewcommand{\theequation}{S\arabic{equation}}
\newcommand{\bd}{\bm}
             %  but any date may be explicitly specified
%
%\pacs{74.20.-z, 74.20.Rp, 74.25.Jb, 74.70.Xa}
             
\begin{abstract}
\begin{itemize}
 \item build up theory for magnon-phonon interactions in localized Heisenberg magnets
 \item discuss special cases and limitations
 \item application to a triangular antiferromagnet
 \item calculate expected cross section for neutron scattering experiments, compare to experimental results
\end{itemize}

\end{abstract}
%\textit{Experimental details [Samples, data analysis].}\\

\keywords{multiferroics, spin-wave, inelastic neutron scattering}
\maketitle
%\textit{Magnons in YMnO$_3$.}\\
\section{Introduction}

\section{Spin-wave expansion}
As discussed in the main text, we start from a localized spin Hamiltonian with spin $S=2$ operators
${\bf S}_{i\bf R}$ located at the positions of the Mn atoms\cite{Mn_positions} in the P6$_3$cm group with setting
$x=0.315$\cite{Petit2007,Toulouse2014},
\begin{equation}
 H_S=\sum_{ij, \bf R \bf R'}J_{ij,{\bf R}  {\bf R}'} {\bf S}_{i\bf R}\cdot {\bf S}_{j {\bf R}'}+\sum_{i, \bf R}\bigl( {\bf h}_{i}\cdot {\bf S}_{i \bf R}+D S_{i\bf R}^z S_{i\bf R}^z\bigr)\;.\label{eq_HS}
\end{equation}
with in-plane coupling $J_{ij,{\bf R}  {\bf R}'}=J$ if $(i,{\bf R})$ and $(j,{\bf R}')$ are nearest neighbors, and two non-equal out-of plane couplings
$J_{z_1}$ and $J_{z_2}$. The easy plane-anisotropy $D$ forces the spins in the classical ground state into the plane.
A second anisotropy can be written in terms of an effective magnetic field, ${\bf h}_{i}={\bf h}-H{\bf n}_i$ where $\bf h$ is an external magnetic field, $H$ is an easy-axis anisotropy
and ${\bf m}_i$ is the direction of the local magnetization at site $i$. 
For calculation purposes, we use a non-symmetric effective g-tensor $g_i$ and can express the effective magnetic field in terms
of an magnetic induction that is arbitrarily directed parallel to the crystallographic $c$ direction, ${\bf b}={\bf e}_z$, via ${\bf h}_i=g_i{\bf b}$.
Next, the classical groundstate is determined by replacing the spin-operators in the above expression by ${\bf S}_{i\bf R}=S{\bf m}_i$ and parametrization of
the local coordinate system via
\begin{subequations}
 \begin{align}
  \bd m_i&=c_\theta [\cos(\theta_i)\bd e_x +\sin(\theta_i)\bd e_y] +s_\theta \bd e_z \,,\\
    \bd e_i^{(1)}&=\sin(\theta_i)\bd e_x -\cos(\theta_i)\bd e_y\,,\\
    \bd e_i^{(2)}&=s_\theta [\cos(\theta_i)\bd e_x +\sin(\theta_i)\bd e_y]-c_\theta \bd e_z\,,\label{eq_loc_coord_h}
 \end{align}
\end{subequations}
where $s_\theta=\sin\theta$ and $c_\theta=\cos\theta$ and $\theta$ is the canting angle of the spins out of the plane an $\theta_i$ is the local rotation angle of the spins
and the minimization of the resulting energy with respect to the angles $\theta_i$ and $\theta$.
For an ordinary single $\bd Q$ antiferromagnet, it would be given by $\theta_i=\bd Q\cdot R_i$, but in our case the alternating layers
have ordering vectors of different directions, and therefore do not allow such a parametrization.
%and local coordinate systems are defined at each position $i$ such that the classical magnetization
%points towards the ${\bf m}_i$ direction and two spherical unitary vectors ${{\bf e}_i}^\pm$ complete the definition of the coordinate system\cite{Kreisel11,Toth15}.
%The spin-wave expansion can then be made for the general case by use of the Holstein-Primakoff transformation\cite{Holstein40} in conjunction with a Fourier transformation with respect
%to the repeating magnetic elementary cell.
Introduction of the
spherical vectors $\bd e_i^{\pm}=\bd e_i^{ (1)}\pm i\bd e_i^{(2)}$
allows to rewrite this rotantion as
 \begin{equation}
  {\bf S}_{i\bf R}=S_{i\bf R}^\parallel \bd m_i  +\bd S_{i\bf R}^\perp=S_{i\bf R}^\parallel \bd m_i+\frac 12 \sum_{p=\pm}S_{i\bf R}^{-p} \bd e_i^{\prime p}.
 \end{equation}
With the Holstein-Primakoff transformation (up to leading order in $1/S$)
\begin{subequations}
  \begin{align}
S_{i\bf R}^+\approx\sqrt{2S}b_{i\bf R}\,,\notag\\
S_{i\bf R}^-\approx\sqrt{2S} b_{i\bf R}^\dagger\,,\notag\\
S_{i\bf R}^\parallel=S-b_{i\bf R}^\dagger b_{i\bf R}\,,\label{eq_HP}
  \end{align}
  \end{subequations}
we obtain the form
\begin{equation}
 {\bf S}_{i\bf R}=\sqrt{\frac S2}\bigl(\bd e_i^{-} b_{i\bf R}+\bd e_i^+b_{i\bf R}^\dagger\bigr)+\bd m_i (S-b_{i\bf R}^\dagger b_{i\bf R})\label{eq_hp_rotation}
\end{equation}
such that all coefficients for the magnon operators $b_{i\bf R}$ in the quadratic Hamiltonian can be collected straight forward\cite{Toth15} and transformed to momentum space.


\section{Phonon Hamiltonian}
\section{Phonon modes in a triangular lattice}
To calculate the phonon modes from a simple model, we use a triangular latice of Mn atoms (for simplicity located at the
ideal $x=1/3$ positions) with lattice constant $a$, mass $m$ and the positions of the
corresponding Bravais lattice $\{ \bd{R}_i \}$ coupled to their nearest neighbors with a spring constant $C$.
Defining the direction vectors  in the plane with the cartesian coordinates
\begin{equation}
\bd a_i=(\cos(\pi(i-1)/3),\sin(\pi(i-1)/3))^T\label{eq_NNvectors}
\end{equation}
we can write down the equations of motion
in frequency space as
\begin{align}
 \omega^2 \frac{m}{a^2 C} \bd u^{(n)}&=\bd a_1 \bd a_1^T(2\bd u^{(n)}-\bd u  ^{(n+e_1)}-\bd u^{(n-e_1)})\notag\\
 &+\bd a_2 \bd a_2^T(2\bd u^{(n)}- \bd u^{(n+e_2)}-\bd u^{(n-e_2)})\notag\\
  &+\bd a_3 \bd a_3^T(2\bd u^{(n)}- \bd u^{(n+e_1-e2)}-\bd u^{(n+e_1-e_2,2)})\,,
\end{align}
Note that $\bd u^{(n)}$ labels the position of the elementary cell $n$.
%Defining the lattice vectors $\bd x=(a,0)^T$ and $\bd y=(0,a\sqrt(3)/2)$, we can rewrite
%$\bd u^{(n+e_s,i)}=\bd u^{(n,i)}e^{i\bd k\cdot \bd s}$ with $s\in\{x,y\}$ as well as for the sum
%$\bd u^{(n+e_s+e_v,i)}=\bd u^{(n,i)}e^{i\bd k\cdot( \bd s+\bd v)}$.
Using the Bloch theorem $u^{(n+ \bd{R}_i)}=u^{(n)} e^{-i\bd k\cdot  \bd{R}_i }$, we rewrite the r.h.s. of the equation above and work out the products
\begin{subequations}
 \begin{align}
  \bd a_1 \bd a_1^T=\frac{1} 4\left(\begin{array}{cc}
                           3&\sqrt 3\\
                           \sqrt 3&1
                          \end{array}\right)\\
  \bd a_2 \bd a_2^T=\frac 14 \left(\begin{array}{cc}
                           0&0\\
                           0&1
                          \end{array}\right)\\
  \bd a_3 \bd a_3^T=\frac 14 \left(\begin{array}{cc}
                           1&-\sqrt 3\\
                           -\sqrt 3&3
                          \end{array}\right)\,,
 \end{align}
\end{subequations}
such that we can rewrite the equations of motion as
\begin{subequations}
 \begin{align}
   \omega^2 \frac{m}{Ca^2} \bd u^{(n)}&=D\bd u^{(n)}
 \end{align}
\end{subequations}
with the matrix
\begin{align}
 {D}=\left(\begin{array}{cc}
          3-3\cos\phi_1\cos\phi_2 &\sqrt{3}\sin \phi_1\sin\phi_2\\
          \sqrt{3}\sin \phi_1\sin\phi_2&3-2\cos 2\phi_2 -\cos\phi_1\cos\phi_2
         \end{array}\right)
\end{align}
where $\phi_1=\sqrt 3/2 k_x a$ and $\phi_2=1/2 k_y a$. The solution of the equation of motion are the eigenfrequencies $w_{\bd{k},s}$ and the corresponding
eigenvectors $\bd e_{\bd{k},_s}$ such that
 \begin{figure}[tb]
%  \includegraphics[width=\linewidth]{phonon_hexagonal-crop}
 \caption{Phonon dispersion plotted along high symmetry directions\cite{Martinsson03}}
 \label{fig_phonon_spectrum_hex}
 \end{figure}
we can write down the phonon Hamiltonian as
 \begin{equation}
{H}_{L} =   \sum_{ \bd{k}  s } \omega_{\bd{k},s } \left( a^{\dagger}_{ \bd{k}, s}
 a_{\bd{k}, s} + \frac{1}{2} \right),
 \label{eq:Hp}
 \end{equation}
where $a_{\bd{k},s}$ annihilates a phonon with wave-vector $\bd{k}$ and
polarization $s$.
In our sytem, the phonon modes are modeled as three acoustic modes $s=1,2,3$, two of them obtained from the two
dimensional system as discussed above, the third obtained from a rotation of the polarization vector ${\bf e}_{\bd k,s}$ of the transversal mode out of the plane by keeping
the eigenenergy degenerate.
We now have for the dynamic positions of the atoms in the full elementary cell ${\bd{r}}_i = \bd{R}_i + \bd{X}_i$.
The lattice distortions are quantized in the usual way,
 \begin{subequations}
 \begin{align}
 \bd{X}_i & =  \frac{1}{\sqrt{N}} \sum_{\bd{k}} e^{ i \bd{k} \cdot \bd{R}_i } 
 \bd{X}_{\bd{k}},
 \\
 \bd{X}_{\bd{k}} & =  \sum_{\lambda} X_{ \bd{k} \lambda } \bd{e}_{ \bd{k} \lambda } ,
 \\
  X_{ \bd{k} \lambda } & =   \frac{1}{ \sqrt{2 M \omega_{\bd{k} \lambda }} }
 ( a_{ \bd{k} \lambda } +  a_{ -\bd{k} \lambda }^{\dagger} ).
 \label{eq_phononop}
 \end{align}
\end{subequations}
Introducing the momentum operators conjugate to the $X_{\bd{k} \lambda}$,
 \begin{equation}
 P_{ \bd{k} \lambda }  =   \frac{1}{i} \sqrt{ \frac{  M \omega_{\bd{k} \lambda }}{2}  }
 ( a_{ \bd{k} \lambda } -  a_{ -\bd{k} \lambda }^{\dagger} ),
\end{equation}
the pure phonon Hamiltonian (\ref{eq:Hp}) can also be written as
 \begin{equation}
{H}^{\mathrm{ pho}} =    \sum_{ \bd{k}  \lambda } \left[
 \frac{ P_{ - \bd{k} \lambda} P_{  \bd{k} \lambda}}{2 M} +
 \frac{M}{2} \omega^2_{ \bd{k} \lambda}  X_{ - \bd{k} \lambda} X_{  \bd{k} \lambda}
 \right].
 \end{equation}


\section{Spin-lattice interactions.}

\section{Magnon-Phonon interaction}
The magnon-phonon interaction can originate from different terms as discussed for example in Ref. \cite{Boiteux72}:
The volume magnetostriction is described by the modulation of the exchange interactions $J_{ij}$
in the presence of phonons. In many magnets this mechanism doesn't play a role since the corresponding
coupling vertex between magnons and phonons vanishes, for example in the ferromagnet\cite{Rueckriegel14} or for
any collinear antiferromagnet.\footnote{We don't go into details here, but the general reason is that to linear order in the
derivative of the exchange coupling with respect to lattice deformations one bond gets stronger and the corresponding bond in
the opposite direction gets weaker by the same amount, so the net effect to the energy vanishes. This is of course not true
any more once interactions with multiple phonons play a role.}
For our spiral magnet, the volume magnetostriction in principle gives a finite contribution\cite{Kreisel11},
but only for couplings to phonons that actually modulate the positions of the ions in the plane.
Experimentally, it has been shown that the strong magnetoelastic coupling arises from
a transverse phonon that has polarization along the crystallographic $c$-axis of YMnO$_3$\cite{Pailhes09,Petit2007}
such that also in our case the coupling from this mechanism will be zero to leading order because
the lattice vibrations simply don't modulate exchange couplings in plane.
For completeness, we will mention the corresponding analysis that has been carried
out in Ref. \cite{Kim07} where the magnon-phonon hybridization of one magnon mode with one phonon has been discussed.
The coupling of magnons and phonons via the crystalline field can be described by
the Hamiltonian\cite{Laurence73,Boiteux72}
\begin{equation}
 H_{\text{SL}}=\sum_i \sum_{\alpha\beta\gamma\delta} G_{\alpha\beta\gamma\delta} S^\alpha_iS^\beta_i \epsilon_{\gamma\delta}^i
\end{equation}
where $G$ is the spin-phonon coupling tensor.
This can be written in terms of irreducible representations of the spin and lattice functions in the hexagonal symmetry class as\cite{Callen65}
\begin{equation}
 H_{\text{SL}}=-\sum_{i}\sum_{\Gamma}\sum_{jj'} \tilde B_{jj'}^\Gamma(i) \sum_{i'}\epsilon_\Gamma^{i'} S_{\Gamma j}^{i'}(i)\,.
\end{equation}
Assuming equivalent ions, we drop the dependence on $j$ and $j'$, e.g. $\tilde B_{jj'}^\Gamma(i) \rightarrow B_\Gamma(f)$
and by translational symmetry also the dependence on $i$.
For the hexagonal system, this reads explicitly
\begin{align}
 H_{\text{SL}}&=-\sum_{i}\biggl( B_{12}^\alpha \epsilon^{\alpha,1}\bigl[(S_i^x)^2+(S_i^y)^2+(S_i^z)^2\bigr]\notag\\
 &+B_{22}^\alpha \epsilon^{\alpha,2} \frac{\sqrt{3}}{2} \bigl[(S_i^z)^2-\frac 13 \bd S^2\bigr]\notag\\
 &+B^\gamma \bigl[\epsilon^{\gamma,1} \frac 12 \bigl[(S_i^x)^2-(S_i^y)^2\bigr]+\epsilon^{\gamma,2} \frac 12 \{S_i^x,S_i^y\}\bigr]\notag\\
  &+B^\epsilon \bigl[\epsilon^{\epsilon,1} \frac 12 \{S_i^y,S_i^z\}+\epsilon^{\epsilon,2} \frac 12 \{S_i^x,S_i^z\}\bigr]\biggr)\label{eq_sl}
\end{align}
where the strains are defined as
\begin{subequations}
 \begin{align}
  \epsilon^{\alpha,1}&=\epsilon_{xx}+\epsilon_{yy}+\epsilon_{zz}\\
  \epsilon^{\alpha,2}&=\frac{\sqrt{3}}{2}\bigl[\epsilon_{zz}-\frac 1 3 \epsilon^{\alpha,1}\bigr]\\
  \epsilon^{\gamma,1}&=\frac 12 \bigl[\epsilon_{xx}-\epsilon_{yy}\bigr]\\
  \epsilon^{\gamma,2}&=\epsilon_{xy}\\
  \epsilon^{\epsilon,1}&=\epsilon_{yz}\\
  \epsilon^{\epsilon,2}&=\epsilon_{xz}\,
 \end{align}
\end{subequations}
where $\{A,B\}=AB+BA$ is the anti commutator and the strain tensor is defined as
\begin{equation}
 \epsilon_{\alpha\beta}^i=\frac 12 (E_{\alpha\beta}+E_{\beta\alpha})=\frac 12 \biggl(\frac{\partial X_i^\beta}{\partial r_\alpha}+\frac{\partial X_i^\alpha}{\partial r_\beta}\biggr).
 \label{eq_strain}
\end{equation}
Since the spin-operators commute with the strain tensor elements, it is possible to write Eq. (\ref{eq_sl}) into a matrix form when only considering the leading order in $1/S$\footnote{In principle it is also possible to approximately take into account higher order corrections from the commutation of the spin operators in the bosonic prescription as discussed in \cite{Jensen_risoe}. At this point we will not include these corrections as they will approximately renormalize the magnon-phonon interaction constants which are unknown parameters at this point.}
\begin{equation}
 H_{SL}=-\sum_{i\bf R} {\bf S}_{i\bf R}^T  E_i {\bf S}_{i\bf R}
\end{equation}
and use Eq. (\ref{eq_hp_rotation}). Retaining only terms that couple linearly similar to the approach of Ref. \onlinecite{Petit2007}, we obtain
\begin{align}
  H_{SL}&=-S\sqrt{\frac S2} \sum_{i\bf R} \left\{(\bd e_i^{-T} b_{i\bf R}+e_i^{+T}b_{i\bf R}^\dagger)E_i \bd m_i \right.\\
  &\left. +\bd m_i^T E_i (\bd e_i^{-} b_{i\bf R}+e_i^{+}b_{i\bf R}^\dagger\right\}
\end{align}
or similarily in momentum space
\begin{align}
  H_{SL}&=-S\sqrt{\frac S2} \sum_{i\bd k} \left\{(\bd e_i^{-T} b_{i\bd k}+e_i^{+T}b_{i\bd k}^\dagger) E_{i,\bd k} \bd m_i \right.\\
  &\left. +\bd m_i^T E_{i,\bd k} (\bd e_i^{-} b_{i\bd k}+e_i^{+}b_{i\bd k}^\dagger\right\}
\end{align}

The nonlocal contributions of the strain tensor can be obtained by a method introduced in Ref. \cite{Evenson69}
where the local strain is replaced by a nearest neighbor contraction
\begin{equation}
\epsilon_{\alpha\beta}^i \rightarrow \tilde\epsilon_{\alpha\beta}^i=\frac 1 n \sum_{\bf \delta} \epsilon_{\alpha\beta}(i,i+\bf \delta)
\end{equation}
where ${\bf \delta}$ is the sum over nearest neighbors and $n$ is a normalization constant that ensures that
$\tilde\epsilon_{\alpha\beta}^i$ reduces to $\epsilon_{\alpha\beta}^i$ in the long-wavelength limit\cite{Jensen_risoe}.
Defining functions that are isomorphic to the strain tensors,
\begin{align}
 \epsilon_{\alpha\beta}(i,j)=\frac 12 &\left[(R_i^\alpha-R_j^\alpha)(X_i^\beta-X_j^\beta)\right.\notag\\
 &\left.+(R_i^\beta-R_j^\beta)(X_i^\alpha-X_j^\alpha)\right]
\end{align}
where $X_i^\alpha$ are the displacements due to lattice vibrations and $R_i^\alpha$ are the cartesian components of the lattice positions of the ions.
With the same set of vectors connecting nearest neighbors as used above, Eq.(\ref{eq_NNvectors}), we can obtain their cartesian components and add
up the contributions after introduction of the phonon operators in momentum space, $X_{\bd k,s}^\alpha$. The normalization constant is simply obtained
by fixing the small $\bd k$ expansion to the result from the derivative in Eq. (\ref{eq_strain}).
For the triangular lattice, we obtain the following structure function $\bd g(\bd k)=(g^{x},g^{y},g^{z})$
\begin{subequations}
 \begin{align}
 g^{x}&=\frac 1{2a}\sin\biggl(\frac h 2\biggr)\cos\biggl(\frac h 6+\frac k 3\biggr)\\
 g^{y}&=\frac 1{2\sqrt{3}a}\biggl[\sin\biggl(\frac{h}{6}+\frac{k}{3}\biggr)\cos\biggl(\frac{h}{2}\biggr)+\sin\biggl(\frac{h}{3}+\frac{2k}{3}\biggr)\biggr]\\
     g^{z}&=\frac 1 c \sin l
  \end{align}
\end{subequations}
where the momentum $(hkl)$ is already expressed with respect to the relevant magnetic elementary cell such that the components of the matrix $E_i$ are given by
\begin{equation}
 E_i^{\alpha\beta}=\frac{1}{N}\sum_{\bd k, s} f_{\bd k,s}^{\alpha \beta} e^{i\bd k\cdot \bd R_i}
\end{equation}
with
\begin{equation}
 f_{\bd k,s}^{\alpha \beta} =\frac i2 g_{\bd k,s}^{\alpha \beta}\frac{a_{\bd k,s}+a_{\bd k,s}^\dagger}{\sqrt{2mw_{\bd k,s}}}
\end{equation}
and the coupling constants $g_{\bd k,s}^{\alpha \beta}$ are sums of products of $B^\sigma$ from the spin-lattice Hamiltonian and momentum-dependent structure 
function and the phonon polarization $\bd g\cdot \bd e_{\bd k,s}$.
Writing
\begin{equation}
 E_{i,\bd k}^{\alpha\beta}=\frac{4}{iS\sqrt S}\sum_s(a_{\bd k,s}+a_{\bd k,s}^\dagger)G_{\bd k,s,i}
\end{equation}
we can see that the spin-lattice Hamiltonian is a hybridization term between the Holstein-Primakoff magnon operators and the phonon
operators with a product $G_{\bd k,s,i}$ and the unitary vectors in Eq. (\ref{eq_loc_coord_h}) as matrix elements.
Rearranging all terms will give the result
\begin{equation}
H= \sum_{\bf k}(\vec b_{\bf k}^\dagger,\vec b_{-\bf k})\mathcal D_{\bf k}\left(\begin{array}{c}
                                                                               \vec b_{\bf k}\\
                                                                               \vec b_{\bf -k}^\dagger
                                                                              \end{array}\right) \label{eq_Htot}\,.
\end{equation}
For the calculation of the ground state and the Fourier transforms of the terms from the spin Hamiltonian, we use the SpinW\cite{Toth15} package, and therefore
follow the notation for the matrices $A(\bd k)$, $B(\bd k)$ and $C$ of that reference.
The grand dynamical matrix is then given by
\begin{equation}
 \mathcal D_{\bf k}=\left(
 \begin{array}{cccc}
  A(\bd k)-C & \Gamma(\bd k) & B(\bd k) &\Gamma(\bd k)\\
  \Gamma^\dagger(\bd k) & W(\bd k) & \Omega(\bd k)&0\\
  B^\dagger(\bd k)& \Omega^\dagger(\bd k) &\bar A(-\bd k)-C &\Omega(\bd k)\\
  \Gamma^\dagger(\bd k) &0&\Omega^\dagger(\bd k) &W(-\bd k)
 \end{array}\right)\,,
\end{equation}
where the phonon dispersions appear in $W(\bd k)=\text{diag}(\{w_{\bd k,s}\})$ and the elements of the matrices are given by
\begin{subequations}
 \begin{align}
 \Gamma(\bd k)^{is}&=\bd e_i^{-T}G_{\bd k,s,i}\bd m_i\\
 \Omega(\bd k)^{is}&=\bd m_i^TG_{-\bd k,s,i}\bd e_i^{-}
 \end{align}
\end{subequations}




\section{Magnetoelastic waves}
Looking at the matrix ${\mathcal D}_{\bf k}$, one sees that not all modes couple to the phonon mode. This is arising from the fact that the single-ion magnetostriction
is uniform in space and only the spiral parts of the Hamiltonian couple to it. Following Colpa \cite{Colpa78,Serga12} we use the algorithm to diagonalize
the Bosonic Hamiltonian giving the para-unitary matrix
\begin{equation}
  {\mathcal J}_{\bd k}=\left(\begin{array}{cc} {\mathbf U}_{\bd k}^\dagger&- {\mathbf V}_{\bd k}^\dagger\\
- {\mathbf W}_{\bd k}^\dagger& {\mathbf X}_{\bd k}^\dagger\end{array}\right)\quad   {\mathcal J}_{\bd k}^{-1}=\left(\begin{array}{cc} {\mathbf U}_{\bd k}& {\mathbf W}_{\bd k}\\
 {\mathbf V}_{\bd k}& {\mathbf X}_{\bd k}\end{array}\right)
\end{equation}
to diagonalize the Hamiltonian (\ref{eq_Htot}).
This transformation connects the vectors of the true magnetoelastic operators $\vec{\gamma}_{\bf k}^\dagger$ to those in the Holstein-Primakoff basis via
\begin{equation}
        \vec{\gamma}_{\bd k}^\dagger={\mathcal J}_{\bd k}  \vec b_{\bd k}^\dagger \quad \Leftrightarrow \quad  {\mathcal J}_{\bd k}^{-1}\vec{\gamma}_{\bd k}^\dagger=  \vec b_{\bd k}^\dagger
        \label{eq_genbogoliubov}
\end{equation}
and transforms to the diagonal Hamiltonian
\begin{equation}
 {H}_2=\frac 12 \sum_{\bd k}\vec \gamma_{\bd k}^\dagger{\mathcal E}_{\bd k} \vec \gamma_{\bd k}+E_0^{(2)}
\end{equation}
with the diagonal matrix ${\mathcal E}_{\bd k}=\text{diag}(\{\omega_{\bd k,l}\})$. In other words ${\mathcal J}_{\bd k}$ is the wavefunction of the coupled magnetoelastic waves.
With our choice of the ordering in $\vec b_{\bf k}$, we can split up to the spin and lattice part of the wavefunction by
\begin{equation}
 {\mathcal J}_{\bd k}^{-1}=\left(
  \begin{array}{cccc}
\mathcal N^{\uparrow}\\
\mathcal M^{\uparrow}\\
\mathcal N^{\downarrow}\\
\mathcal M^{\downarrow}
  \end{array}\right)
\end{equation}
and define the matrices $\mathcal N$ and $\mathcal M$ via
\begin{equation}
\mathcal A=\left(
   \begin{array}{cccc}
    \mathcal A^{\uparrow}\\
\mathcal A^{\downarrow}
   \end{array}\right)\,.
\end{equation}



\section{Nuclear structure factor}
In this section we derive the nuclear structure factor in presence of magnetoelastic waves as derived above.
Starting point is the expression for the coherent nuclear cross section in Ref. \onlinecite{Lovesey_book}, chapter 4.4:
%\begin{align}
 %\left.\frac{d^2 \sigma}{d \Omega dE'}\right|_{\text{incoh}}^{\text{inel}}&=\frac{\sigma_i}{4\pi}\frac{k}{k'} \frac{1}{2\pi \hbar}\int_{-\infty}^{\infty} dt e^{-i\omega t} e^{-2W(\bd k)}\notag\\
 %&\times \sum_i\left[e^{\langle \bd k\cdot\bd X_i\,\bd k\cdot\bd X_i(t)\rangle}-1\right]
%\end{align}
%and for the coherent contribution that we will consider in the following
\begin{align}
  \left.\frac{d^2 \sigma}{d \Omega dE'}\right|_{\text{coh}}^{\text{inel}}&=\frac{\sigma_i}{4\pi}\frac{k}{k'} \frac{1}{2\pi \hbar}\int_{-\infty}^{\infty} dt e^{-i\omega t} e^{-2W(\bd k)}\notag\\
 &\times \sum_{ij}e^{i\bd k(\bd R_i-\bd R_j) \langle \bd k\cdot\bd X_i\,\bd k\cdot\bd X_j(t)\rangle}\,.
\end{align}
In case of lattice vibrations only, one simply puts in the time dependence of the phonon operator and simplifies to the expressions given in Eq. (4.70) and (4.71) of Ref. \onlinecite{Lovesey_book},
but for the magnetoelastic waves, we first have to find the correct time-dependence of the phonon operator.
Essentially, we know the time dependence of the magnetoelastic operators
\begin{equation}
 \gamma_l(\bd q,t)^\dagger=\gamma_l(\bd q)e^{i\omega_{\bd q l}t} 
\end{equation}
such that we have to use Eq. \ref{eq_genbogoliubov} to obtain for the phonon operator
\begin{equation}
 a_{\bd q, s}=\sum_l\left([{\mathbf U}_{\bd k}]_{sl}(\bd q)\gamma_l(\bd q)+[{\mathbf W}_{\bd q}]_{sl}\gamma_l(-\bd q)^\dagger\right)
\end{equation}
and consequently the time-dependence
\begin{align}
 a_{\bd q,s}&=\sum_l\left([{\mathbf U}_{\bd q}]^*_{sl}(\bd q)\gamma_l(\bd q)^\dagger e^{i\omega_{\bd q,l}t}\right.\notag\\
 &\left.+[{\mathbf W}_{\bd q}]^*_{sl}\gamma_l(-\bd k)e^{-i\omega_{-\bd q,l}t}\right)\,.
\end{align}
Putting everything together, we can rewrite the expectation value
\begin{align}
 \langle \bd k\cdot\bd X_i\,\bd k\cdot\bd X_j(t)\rangle=\frac{|\bd k\cdot \bd e_{\bd k,s}|^2}{2NM}\sum_{\bd k}\frac{e^{i\bd k\cdot(\bd R_i-\bd R_j)}}{w_{\bd k,s}}\notag\\
 \times \sum_{ls}([\mathbf W_{\bd k}]_{sl}+[\mathbf U_{-\bd k}]^*_{sl})([\mathbf U_{\bd k}]_{sl}+[\mathbf W_{\bd k}]_{sl}^*)\notag\\
 \times\bigl\{[1+n(\omega_{\bd k,l})] e^{i\omega_{\bd k,l}t} +n(\omega_{\bd k,l}) e^{-i\omega_{\bd k,l}t}\bigr\}\,,
\end{align}
where $n(\omega)$ is the Bose function.
Finally, we arrive at the expression
%For simplicity and because it is a good approximation, we take the limit $T\rightarrow 0$ where only the second term contributes giving
\begin{align}
  \left.\frac{d^2 \sigma}{d \Omega dE'}\right|_{\text{coh}}^{\text{inel}}=\frac{\sigma_c}{4\pi}\frac{k'}{k} \frac{(2\pi)^3}{2V_0M} e^{-2W(\bd q)} |\bd k\cdot\bd e_{\bd k,s}|^2\notag\\
  \times \sum_{\bd k,l,s}\frac{\mathcal M_{si}}{mw_{{\bf k}s}} \Delta(\omega,{\bf k},l)\label{eq_nuc_structure}
\end{align}
with
$\Delta(\omega,{\bf k},l)= \delta(\omega- \omega_{{\bf k},l}d_{l,l})[n(\omega) +\frac 12 (1-d_{ll})]$ and $d_{ll}$ is a generalized Kronecker delta being negative for $l>N+M$.
The prefactor containing the Debeye-Waller factor, the nuclear cross section of the corresponding atom and the kinematic factor $k'/k$ is ommitted in the expression of the main text where
the dynamical structure factor is cited, since it is a momentum-independent constant which has to be adjusted to the experimental data.
Note that the kinematic factor is experimentally divided out when accounting for
constant counts at monitors.
%\begin{align}
% S_{\text{nuc}}({\bf q},\omega) =\sum_{s} |{\bf q}\cdot{\bf e}_{\bd k,s}|^2
%  \sum_{l=1}^{2(N+M)}\frac{\mathcal M_{si}}{mw_{{\bf q}s}}\Delta(\omega,{\bf q},l)\, ,\label{eq_nuc_structure}
%\end{align}
%where $\bd G$ is a reciprocal vector and the combined matrix elements are given by
%$M(n,\bd k)=([\mathbf W_{\bd k}]_{4n}+[\mathbf X_{\bd k}]_{4n}])([\mathbf U_{\bd k}]_{4n}+[\mathbf V_{\bd k}]_{4n}])$.
% Note that the Debeye-Waller factor should be approximated by\cite{Lovesey_book}
% \begin{equation}
%  W(\bd q)=\frac{\hbar  q^2}{4M}\int_0^{\omega_m}\,d\omega \frac{Z(\omega)}{\omega}\coth(\frac 12 \hbar \omega \beta)\,;
% \end{equation}
% in the limit $T\rightarrow 0$, we can again simplify $\coth(\frac 12 \hbar \omega \beta)=1$ and rewrite
% \begin{equation}
%   W(\bd q)=\frac{\hbar  q^2}{4M}\int_0^{\omega_m}\,d\omega \frac{Z(\omega)}{\omega}
%   =W_0q^2\,.
% \end{equation}
% In order to calculate the dynamical structure factor, at this point we ignore the prefactors and the Debeye Waller factor
% as well as the sum over $\bd G$ which just brings us back to the first Brillouin zone with $\bd k=\bd q$.
When evaluating the expression above, we broaden the result in energy by a convolution with a Gaussian containing the experimental resolution.
%\begin{equation}
% \delta(\omega-\omega_n(\bd q))=\pi \frac{\eta}{(\omega-\omega_n(\bd q))^2+\eta^2}\,.
%\end{equation}
\section{Magnetic cross section}
For the mangnetic cross section, we follow the standart procedure in calculating the dynamical structure factor from the magnon
operators in the Holstein-Primakoff basis\cite{Lovesey_book}, given the eigenstates of the full Hamiltonian.
The transformations from the local coordinate system defined by Eq. (\ref{eq_loc_coord_h}) as well as the magnon part of the
transformation in Eq. (\ref{eq_genbogoliubov}) has to be inserted, with the result\cite{Toth15}
\begin{equation}
 { S}^{\alpha\beta}_{\text{mag}}({\bf q},\omega)=\sum_{l=1}^{2(N+M)}\left[\mathcal N^\dagger e^{\alpha\beta}({\bf k}) \mathcal N\right]_{ll} \Delta(\omega,{\bf q},l)\, .
\end{equation}
The total magnetic cross section is then given by kinematic prefactors, the magnetic cross section $r_0$, the magnetic form factor $f(\bd q)$ and transverse spin structure factor,
\begin{align}
  \left.\frac{d^2 \sigma}{d \Omega dE'}\right|_{\text{coh}}=&r_0^2 \frac{k'}{k} (gf(\bd q))^2e^{-2W(\bd q)}\notag\\
  &\times \sum_{\alpha\beta}(\delta_{\alpha\beta}-\frac{q_\alpha q_\beta}{\bd q^2}){ S}^{\alpha\beta}_{\text{mag}}({\bf q},\omega)
\end{align}
such that we can use the expression
\begin{align}
  \left.\frac{d^2 \sigma}{d \Omega dE'}\right|_{\text{coh}}=P_{\text{mag}}f(\bd q)^2\sum_{\alpha\beta}(\delta_{\alpha\beta}-\frac{q_\alpha q_\beta}{\bd q^2}){ S}^{\alpha\beta}_{\text{mag}}({\bf q},\omega)
\end{align}
for the calculation of the magnetic contribution to the measured count rates, where we use the magnetic form factor of Mn$^{2+}$\cite{TabCrys}
and a experiment specific prefactor $P_{\text{mag}}$.


\section{Comparison to experimental data}
In order to reproduce the structure of magnon excitations in the magnetic field, it is required to take into account all 6 magnetic ions in the elementary cell.
\begin{figure*}[]
%\includegraphics[width=\textwidth]{Maps.png}
\caption{
Color-maps of the measured magnon and phonon dispersions along the main symmetry directions around $q=(100)$ and $q=(300)$. The data has been taken on EIGER and the small high resolution corners have been measured at \mbox{RITA-2}. An   overview of the $q$-positions used for the energy scans are shown in figure \ref{fig:qspace}. At $q=(100)$ the magnon signal is strong due to the magnetic structure factor. At $q=(300)$ the phonon and magnon intensities are similar as   the structure factors match at this $q$-position.
Top left: Phonon dispersion at $q=(300)$ measured above $T_N$ ($T=100$ K). Top right: Magnon dispersion at $q=(100)$ measured at base temperature ($T=1.6$ K). Bottom left: Phonon and magnon dispersions measured at $q=(300)$ and 1.6 K.   Bottom right: Magnon dispersions at $q=(100)$ measured in an applied magnetic field of 13 T and at $T=1.6$ K.
\label{fig:maps}}
\end{figure*}


\begin{figure*}[]
%\includegraphics[width=\textwidth]{{YMnO_S_nuc_plot_8_h_13_dim_3_mode_mK3bsummed7_0607-nup}}
\caption{Simulated neutron intensity for the scans as performed experimentally, for clarity, the broadening is fixed to the experimental broadening of the \mbox{RITA-2} experiment, except for the phonon map at high temperatures (top     left).}
\label{fig_maps_calc}
\end{figure*}

\section{Summary and Conclusions}



\bibliographystyle{phd_mypub}
% \bibliographystyle{apsrev4-1}
\bibliography{YMnO}{}
\end{document}
%
% ****** End of file apssamp.tex ******